% !TEX program = pdflatex
% !TEX encoding = UTF-8 Unicode

%====================================================================
% Robust LaTeX Template for Journal Draft Submissions
%
% This template is designed to be robust and flexible for various
% journal requirements. It includes:
%   - Two title pages (full and blinded)
%   - ORCID support
%   - Flexible author/affiliation management
%   - Standard journal features
%   - Error-resistant structure
%   - Lorem ipsum content for demonstration
%====================================================================

%--------------------------------------------------------------------
% Document Class
%--------------------------------------------------------------------
\documentclass[12pt, letterpaper, onecolumn, final]{article}

%--------------------------------------------------------------------
% Essential Packages
%--------------------------------------------------------------------
\usepackage[utf8]{inputenc}
\usepackage[T1]{fontenc}
\usepackage{amsmath}
\usepackage{amssymb}
\usepackage{amsthm}
\usepackage{graphicx}
\usepackage{booktabs}
\usepackage{array}
\usepackage{tabularx}
\usepackage{longtable}
\usepackage{multirow}
\usepackage{caption}
\usepackage{subcaption}
\usepackage{float}
\usepackage{enumitem}
\usepackage{microtype}
\usepackage{xcolor}
\usepackage{soul}
\usepackage{url}
\usepackage{xspace}
\usepackage{lipsum} % For generating dummy text
\usepackage{threeparttable}

% Page layout
\usepackage[top=1in, bottom=1in, left=1in, right=1in]{geometry}

% Line spacing options
\usepackage{setspace}
% \doublespacing % Uncomment for double spacing
% \onehalfspacing % Uncomment for 1.5 spacing

% Line numbers (often required for review)
\usepackage{lineno}
% \linenumbers % Uncomment to enable line numbers

% Citation management
\usepackage[sort&compress]{natbib}
% Alternative: \usepackage[authoryear,sort]{natbib}

% Hyperlinks (load near end)
\usepackage[
    colorlinks=true,
    linkcolor=black,
    citecolor=blue,
    urlcolor=blue,
    filecolor=blue,
    pdfpagemode=UseNone,
    pdfstartview=FitH,
    pdfcreator={LaTeX},
    pdfproducer={pdflatex}
]{hyperref}

% Cross-referencing (load after hyperref)
\usepackage[capitalize,noabbrev]{cleveref}

%--------------------------------------------------------------------
% Journal-specific customizations
%--------------------------------------------------------------------

% Caption formatting
\captionsetup[figure]{
    font=small,
    labelfont=bf,
    justification=centering,
    singlelinecheck=false
}
\captionsetup[table]{
    font=small,
    labelfont=bf,
    justification=raggedright,
    singlelinecheck=false
}

% Section formatting
\usepackage{titlesec}
\titleformat{\section}{\normalfont\large\bfseries}{\thesection}{1em}{}
\titleformat{\subsection}{\normalfont\normalsize\bfseries}{\thesubsection}{1em}{}
\titleformat{\subsubsection}{\normalfont\normalsize\itshape}{\thesubsubsection}{1em}{}

%--------------------------------------------------------------------
% Document Information Variables
%--------------------------------------------------------------------
% Define manuscript information
\newcommand{\manuscripttitle}{Advanced Computational Methods for Complex Systems Analysis}
\newcommand{\manuscriptsubtitle}{A Novel Framework for Data-Driven Decision Making}
\newcommand{\runninghead}{Computational Methods for Complex Systems}

% Author information
\newcommand{\authorone}{John A. Smith}
\newcommand{\authoroneaffil}{1,2}
\newcommand{\authoroneemail}{john.smith@university.edu}
\newcommand{\authoroneorcid}{0000-0000-0000-0000}

\newcommand{\authortwo}{Jane B. Doe}
\newcommand{\authortwoaffil}{1}
\newcommand{\authortwoemail}{jane.doe@university.edu}
\newcommand{\authortwoorcid}{0000-0000-0000-0001}

\newcommand{\authorthree}{Robert C. Johnson}
\newcommand{\authorthreeaffil}{2,3}
\newcommand{\authorthreeemail}{robert.johnson@institute.org}
\newcommand{\authorthreeorcid}{0000-0000-0000-0002}

% Affiliations
\newcommand{\affilone}{Department of Computer Science, University of Excellence, Excellence City, State, USA}
\newcommand{\affiltwo}{Institute for Computational Research, Excellence City, State, USA}
\newcommand{\affilthree}{National Laboratory for Advanced Computing, Research City, State, USA}

% Corresponding author
\newcommand{\correspondingauthor}{\authorthree}
\newcommand{\correspondingemail}{\authoroneemail}
\newcommand{\correspondingaddress}{908 W. Nevada St, Urbana, Illinois 61801}

% Keywords
\newcommand{\manuscriptkeywords}{computational methods, complex systems, data analysis, machine learning, optimization}

% Abstract
\newcommand{\manuscriptabstract}{%
Complex systems analysis requires sophisticated computational approaches to handle large-scale, multidimensional data efficiently. This paper presents a novel framework that combines machine learning techniques with traditional optimization methods to address challenges in complex systems modeling. Our approach demonstrates significant improvements in computational efficiency and accuracy compared to existing methods. We evaluated the framework using three distinct datasets representing different domains: financial markets, biological networks, and social systems. Results show that our method achieves up to 35\% improvement in processing speed while maintaining 98.7\% accuracy in predictions. The framework's flexibility allows for easy adaptation to various problem domains, making it a valuable tool for researchers and practitioners working with complex systems. These findings contribute to the growing body of knowledge in computational methods and provide a foundation for future research in this rapidly evolving field.%
}

%--------------------------------------------------------------------
% Custom Commands
%--------------------------------------------------------------------

% Common abbreviations
\newcommand{\eg}{e.g.,\xspace}
\newcommand{\ie}{i.e.,\xspace}
\newcommand{\etal}{et al.\xspace}
\newcommand{\vs}{vs.\xspace}
\newcommand{\etc}{etc.\xspace}

% ORCID command
\usepackage{orcidlink}


% Highlight command for drafts
\newcommand{\highlight}[1]{\colorbox{yellow}{#1}}
\newcommand{\todo}[1]{\textcolor{red}{\textbf{TODO: #1}}}

% Units (simple version)
\newcommand{\unit}[1]{\ensuremath{\,\mathrm{#1}}}

%--------------------------------------------------------------------
% Theorem Environments
%--------------------------------------------------------------------
\theoremstyle{plain}
\newtheorem{theorem}{Theorem}[section]
\newtheorem{lemma}[theorem]{Lemma}
\newtheorem{proposition}[theorem]{Proposition}
\newtheorem{corollary}[theorem]{Corollary}

\theoremstyle{definition}
\newtheorem{definition}[theorem]{Definition}
\newtheorem{example}[theorem]{Example}
\newtheorem{remark}[theorem]{Remark}

\theoremstyle{remark}
\newtheorem{note}[theorem]{Note}

%--------------------------------------------------------------------
% Cross-reference names
%--------------------------------------------------------------------
\crefname{section}{Section}{Sections}
\crefname{subsection}{Subsection}{Subsections}
\crefname{figure}{Figure}{Figures}
\crefname{table}{Table}{Tables}
\crefname{equation}{Equation}{Equations}
\crefname{theorem}{Theorem}{Theorems}
\crefname{lemma}{Lemma}{Lemmas}
\crefname{definition}{Definition}{Definitions}
\crefname{example}{Example}{Examples}

%--------------------------------------------------------------------
% Begin Document
%--------------------------------------------------------------------
\begin{document}

% Suppress page numbers for title pages
\pagenumbering{gobble}

%====================================================================
% FIRST TITLE PAGE (Complete Information)
%====================================================================
\begin{titlepage}
    \centering
    
    \includegraphics[width=\textwidth]{Cover_header.png}\\
    \vspace*{0.5cm}
    % Title
    {\LARGE\bfseries \manuscripttitle \par}
    \vspace{0.5cm}
    {\large \manuscriptsubtitle \par}
    \vspace{1.0cm}
    
    % Authors with affiliations
    {\large
    \begin{tabular}{c}
        \authorone\textsuperscript{\authoroneaffil} \\
        \small\authoroneemail \\
        \small\orcidlinkc{\authoroneorcid} \\[0.5cm]
        
        \authortwo\textsuperscript{\authortwoaffil} \\
        \small\authortwoemail \\
        \small\orcidlinkc{\authortwoorcid} \\[0.5cm]
        
        \authorthree\textsuperscript{\authorthreeaffil} \\
        \small\authorthreeemail \\
        \small\orcidlinkc{\authorthreeorcid} \\
    \end{tabular}
    \par}
    
    \vspace{1cm}
    
    % Affiliations
    {\small
    \begin{tabular}{@{}p{0.8\textwidth}@{}}
        \textsuperscript{1}\affilone \\[0.3cm]
        \textsuperscript{2}\affiltwo \\[0.3cm]
        \textsuperscript{3}\affilthree \\
    \end{tabular}
    \par}
    
    \vspace{1cm}
    
    % Date
    {\large \today \par}
    
    \vspace{1cm}
    
    \vfill
    % Corresponding author
    {\small \raggedright
    \textbf{Corresponding Author:} \correspondingauthor, \correspondingemail, \correspondingaddress
    \par}
    ~\\
    \includegraphics[width=\textwidth]{Cover_footer.png}\\
    
    
    
    % Abstract
    \begin{minipage}{0.8\textwidth}
        \begin{center}
            \textbf{ABSTRACT}
        \end{center}
        \vspace{0.5cm}
        \manuscriptabstract
        
        \vspace{1cm}
        \noindent\textbf{Keywords:} \manuscriptkeywords
    \end{minipage}
    
    \vfill
    
    % Running head information
    %{\small \textbf{Running Head:} \runninghead}
        
\end{titlepage}

%====================================================================
% SECOND TITLE PAGE (Blinded for Review)
%====================================================================
\newpage
\begin{titlepage}
    \centering
    \vspace*{2cm}
    
    % Title only
    {\LARGE\bfseries \manuscripttitle \par}
    \vspace{0.5cm}
    {\large \manuscriptsubtitle \par}
    
    \vspace{3cm}
    
    % Abstract
    \begin{minipage}{0.8\textwidth}
        \begin{center}
            \textbf{ABSTRACT}
        \end{center}
        \vspace{0.5cm}
        \manuscriptabstract
        
        \vspace{1cm}
        \noindent\textbf{Keywords:} \manuscriptkeywords
    \end{minipage}
    
    \vfill
    
    % Date
    {\large \today \par}
    
\end{titlepage}

% Resume page numbering
\newpage
\pagenumbering{arabic}
\setcounter{page}{1}

%====================================================================
% TABLE OF CONTENTS (Optional)
%====================================================================
% \tableofcontents
% \newpage

%====================================================================
% MAIN CONTENT
%====================================================================

\section{Introduction}
\label{sec:introduction}

\lipsum[1-2] The field of complex systems analysis has experienced rapid growth in recent years, driven by the increasing availability of large-scale datasets and computational resources. \lipsum[3]

Previous studies have shown \citep{example2023, another2022} that traditional approaches often fail to capture the intricate relationships within complex systems. \lipsum[4] Our work builds upon these foundations while addressing the critical limitations identified in earlier research.

\subsection{Research Objectives}
\label{sec:objectives}

The primary objectives of this study are:
\begin{enumerate}
    \item To develop a novel computational framework for complex systems analysis
    \item To demonstrate improved efficiency and accuracy compared to existing methods
    \item To validate the approach across multiple domains and datasets
    \item To provide guidelines for practical implementation
\end{enumerate}

\lipsum[5]

\subsection{Contribution and Significance}
\label{sec:contribution}

\lipsum[6] This research makes several important contributions to the field:

\begin{itemize}
    \item A unified framework that combines multiple computational approaches
    \item Empirical validation across three distinct domains
    \item Open-source implementation for reproducibility
    \item Comprehensive performance benchmarking
\end{itemize}

\lipsum[7]

\section{Literature Review}
\label{sec:literature}

\lipsum[8-9] The literature on complex systems analysis can be broadly categorized into several key areas: traditional statistical methods, machine learning approaches, and hybrid techniques.

\subsection{Traditional Statistical Methods}
\label{sec:traditional-methods}

\lipsum[10-11] Classical statistical approaches have been the foundation of complex systems analysis for decades. \citep{foundational2020} However, these methods often struggle with high-dimensional data and non-linear relationships.

\subsection{Machine Learning Approaches}
\label{sec:ml-approaches}

\lipsum[12] Recent advances in machine learning have opened new possibilities for complex systems analysis. \lipsum[13-14]

\subsection{Hybrid Techniques}
\label{sec:hybrid}

\lipsum[15] The combination of traditional and modern approaches has shown promising results in various applications. \lipsum[16]

\section{Theoretical Framework}
\label{sec:theory}

\lipsum[17-18] Our theoretical framework is based on the integration of optimization theory and machine learning principles.

\begin{definition}
A complex system $S$ can be represented as a tuple $(V, E, F)$ where $V$ is the set of variables, $E$ represents the relationships between variables, and $F$ is the set of functions describing system dynamics.
\end{definition}

\lipsum[19]

\begin{theorem}
Let $S = (V, E, F)$ be a complex system. If the system satisfies certain regularity conditions, then our framework converges to an optimal solution with probability $1-\delta$ where $\delta < 0.05$.
\end{theorem}

\begin{proof}
The proof follows from the convergence properties of the underlying optimization algorithm and the consistency of the machine learning estimators. \lipsum[20]
\end{proof}

\section{Methodology}
\label{sec:methodology}

\lipsum[21-22] Our methodology consists of four main phases: data preprocessing, feature extraction, model training, and validation.

\subsection{Data Preprocessing}
\label{sec:preprocessing}

\lipsum[23] The preprocessing phase involves several steps:

\begin{enumerate}
    \item Data cleaning and outlier detection
    \item Normalization and scaling
    \item Missing value imputation
    \item Feature selection
\end{enumerate}

\lipsum[24]

\subsection{Feature Extraction}
\label{sec:feature-extraction}

\lipsum[25-26] Feature extraction is performed using a combination of statistical and machine learning techniques.

\subsection{Model Architecture}
\label{sec:architecture}

The core of our framework is based on the following optimization problem:

\begin{equation}
    \min_{\theta} \mathcal{L}(\theta) = \frac{1}{n} \sum_{i=1}^{n} \ell(f_{\theta}(x_i), y_i) + \lambda R(\theta)
    \label{eq:objective}
\end{equation}

where $\mathcal{L}(\theta)$ is the loss function, $\ell$ is the individual loss, $f_{\theta}$ is our model parameterized by $\theta$, and $R(\theta)$ is a regularization term with parameter $\lambda$.

\lipsum[27]

\subsection{Training Procedure}
\label{sec:training}

\lipsum[28-29] The training procedure uses an adaptive learning rate schedule and early stopping to prevent overfitting.

The convergence criterion is defined as:
\begin{equation}
    |\mathcal{L}^{(t)} - \mathcal{L}^{(t-1)}| < \epsilon
    \label{eq:convergence}
\end{equation}

where $\epsilon = 10^{-6}$ in our experiments.

\section{Experimental Setup}
\label{sec:experiments}

\lipsum[30] We evaluated our framework using three distinct datasets representing different domains.

\subsection{Datasets}
\label{sec:datasets}

\lipsum[31]

\begin{itemize}
    \item \textbf{Financial Dataset:} Daily stock prices and trading volumes for 500 companies over 10 years
    \item \textbf{Biological Dataset:} Gene expression data from 1,000 samples across 20,000 genes
    \item \textbf{Social Network Dataset:} User interactions and content sharing patterns from a major social platform
\end{itemize}

\lipsum[32]

\subsection{Evaluation Metrics}
\label{sec:metrics}

\lipsum[33] We used several metrics to evaluate performance:

\begin{itemize}
    \item Accuracy and precision for classification tasks
    \item Mean squared error (MSE) for regression tasks
    \item Computational time and memory usage
    \item Scalability across different dataset sizes
\end{itemize}

\subsection{Baseline Methods}
\label{sec:baselines}

\lipsum[34] We compared our approach against several established methods:

\begin{enumerate}
    \item Traditional statistical regression
    \item Support vector machines
    \item Random forests
    \item Deep neural networks
    \item Ensemble methods
\end{enumerate}

\section{Results}
\label{sec:results}

\lipsum[35-36] Our experimental results demonstrate significant improvements across all evaluation metrics.

\subsection{Performance Comparison}
\label{sec:performance}

\cref{tab:performance} summarizes the performance comparison across different methods and datasets.

\begin{table}[htbp]
    \centering
    \caption{Performance comparison across methods and datasets}
    \label{tab:performance}
    \begin{tabular}{lcccc}
        \toprule
        Method & Financial & Biological & Social & Average \\
        \midrule
        Traditional Regression & 0.72 & 0.68 & 0.71 & 0.70 \\
        Support Vector Machine & 0.78 & 0.74 & 0.76 & 0.76 \\
        Random Forest & 0.81 & 0.79 & 0.80 & 0.80 \\
        Deep Neural Network & 0.85 & 0.82 & 0.84 & 0.84 \\
        Ensemble Methods & 0.87 & 0.85 & 0.86 & 0.86 \\
        \textbf{Our Method} & \textbf{0.94} & \textbf{0.91} & \textbf{0.93} & \textbf{0.93} \\
        \bottomrule
    \end{tabular}
    \begin{tablenotes}
        \small
        \item Note: Values represent accuracy scores. Best performance in bold.
    \end{tablenotes}
\end{table}

\lipsum[37]

\subsection{Computational Efficiency}
\label{sec:efficiency}

\cref{fig:efficiency} shows the computational time comparison across different methods.

\begin{figure}[htbp]
    \centering
    \includegraphics[width=0.8\textwidth]{example-image-a}
    \caption{Computational time comparison across methods. Our approach demonstrates superior efficiency, especially for large datasets. Error bars represent standard deviation across 10 independent runs.}
    \label{fig:efficiency}
\end{figure}

\lipsum[38] The results clearly show that our method achieves faster convergence while maintaining high accuracy.

\subsection{Scalability Analysis}
\label{sec:scalability}

\lipsum[39-40] We tested the scalability of our approach using datasets of varying sizes.

\begin{figure}[htbp]
    \centering
    \includegraphics[width=0.8\textwidth]{example-image-b}
    \caption{Scalability analysis showing how computational time scales with dataset size. Our method exhibits linear scaling, making it suitable for large-scale applications.}
    \label{fig:scalability}
\end{figure}

As shown in \cref{fig:scalability}, our method maintains linear scaling even for very large datasets.

\subsection{Sensitivity Analysis}
\label{sec:sensitivity}

\lipsum[41] We conducted a comprehensive sensitivity analysis to understand the impact of various hyperparameters.

\begin{table}[htbp]
    \centering
    \caption{Sensitivity analysis for key hyperparameters}
    \label{tab:sensitivity}
    \begin{tabular}{lccc}
        \toprule
        Parameter & Range & Optimal Value & Sensitivity \\
        \midrule
        Learning Rate & 0.001-0.1 & 0.01 & Medium \\
        Regularization & 0.0001-0.1 & 0.001 & Low \\
        Batch Size & 16-512 & 128 & High \\
        Hidden Layers & 2-10 & 5 & Medium \\
        \bottomrule
    \end{tabular}
\end{table}

\lipsum[42]

\section{Discussion}
\label{sec:discussion}

\lipsum[43-44] The results of our study have several important implications for the field of complex systems analysis.

\subsection{Theoretical Implications}
\label{sec:theoretical-implications}

\lipsum[45] Our findings contribute to the theoretical understanding of complex systems in several ways:

\begin{itemize}
    \item Demonstration of convergence properties under realistic conditions
    \item Characterization of computational complexity
    \item Extension of existing theoretical frameworks
\end{itemize}

\lipsum[46]

\subsection{Practical Applications}
\label{sec:applications}

\lipsum[47-48] The practical implications of our work span multiple domains:

\subsubsection{Financial Markets}
\label{sec:finance-app}

\lipsum[49] In financial applications, our method can be used for risk assessment, portfolio optimization, and market prediction.

\subsubsection{Biological Systems}
\label{sec:bio-app}

\lipsum[50] For biological applications, the framework enables improved analysis of gene networks, protein interactions, and disease mechanisms.

\subsubsection{Social Networks}
\label{sec:social-app}

\lipsum[51] In social network analysis, our approach facilitates better understanding of information propagation, community detection, and influence patterns.

\subsection{Limitations and Future Work}
\label{sec:limitations}

\lipsum[52] While our results are promising, several limitations should be acknowledged:

\begin{enumerate}
    \item Computational requirements for very large datasets
    \item Need for domain-specific parameter tuning
    \item Limited testing on streaming data scenarios
    \item Interpretability challenges in complex models
\end{enumerate}

\lipsum[53]

Future research directions include:
\begin{itemize}
    \item Development of more efficient algorithms for streaming data
    \item Integration with federated learning approaches
    \item Enhancement of model interpretability
    \item Extension to multi-modal data types
\end{itemize}

\section{Conclusion}
\label{sec:conclusion}

\lipsum[54] This study presented a novel computational framework for complex systems analysis that demonstrates significant improvements in both accuracy and efficiency compared to existing methods.

Key contributions include:
\begin{enumerate}
    \item A unified framework combining optimization and machine learning
    \item Comprehensive evaluation across multiple domains
    \item Superior performance in accuracy and computational efficiency
    \item Open-source implementation for community use
\end{enumerate}

\lipsum[55] The results suggest that our approach represents a significant advance in the field and provides a foundation for future research in complex systems analysis.

\section*{Acknowledgments}

The authors thank the National Science Foundation for support through grant \#NSF-12345678. We also acknowledge the High-Performance Computing Center at University of Excellence for providing computational resources, and Dr. Maria Rodriguez for valuable discussions during the early stages of this work.

\section*{Author Contributions}

J.A.S. conceived the study, designed the theoretical framework, implemented the core algorithms, and wrote the first draft. J.B.D. performed the experimental validation, conducted statistical analyses, and created all figures and tables. R.C.J. supervised the project, provided theoretical guidance, and extensively revised the manuscript. All authors contributed to the interpretation of results, participated in regular project meetings, and approved the final version.

\section*{Conflicts of Interest}

The authors declare no conflicts of interest. J.A.S. holds stock in TechCorp Inc., but this relationship did not influence the research design, data collection, analysis, or interpretation of results.

\section*{Funding}
This work was supported by the National Science Foundation [grant numbers NSF-12345678, NSF-87654321]; the Department of Energy [grant number DOE-2024-ABC]; and the Smith Foundation Fellowship awarded to J.A.S.

\section*{Supplementary Material}
Supplementary material for this article is available online at [journal URL]. This includes additional figures, extended derivations, raw data tables, and validation results not included in the main text.


\section*{Data Availability Statement}

The datasets generated and analyzed during the current study are available in the GitHub repository at \url{https://github.com/username/complex-systems-framework}. The financial dataset is subject to licensing restrictions and is available from the corresponding author upon reasonable request and execution of appropriate data use agreements.

\section*{Code Availability}
All code used in this study is available at \url{https://github.com/username/complex-systems-framework} under the MIT license. The repository includes implementation of all algorithms, data processing scripts, and notebooks to reproduce all figures and tables.

\section*{Ethical Approval}
This study was approved by the Institutional Review Board of the University of Excellence (Protocol \#IRB-2024-123). All participants provided informed consent, and all procedures were performed in accordance with relevant guidelines and regulations.


%====================================================================
% APPENDICES
%====================================================================
\appendix

\section{Supplementary Methods}
\label{app:methods}

\lipsum[56-57]

\subsection{Detailed Algorithm Description}
\label{app:algorithm}

\lipsum[58]

Algorithm 1 provides the detailed pseudocode for our main optimization procedure:

\begin{enumerate}
    \item Initialize parameters $\theta_0$ randomly
    \item For $t = 1, 2, \ldots, T$:
    \begin{enumerate}
        \item Compute gradient $\nabla_\theta \mathcal{L}(\theta_t)$
        \item Update parameters: $\theta_{t+1} = \theta_t - \alpha_t \nabla_\theta \mathcal{L}(\theta_t)$
        \item Check convergence criterion \cref{eq:convergence}
        \item If converged, return $\theta_{t+1}$
    \end{enumerate}
    \item Return $\theta_T$
\end{enumerate}

\lipsum[59]

\subsection{Hyperparameter Selection}
\label{app:hyperparams}

\lipsum[60]

\section{Additional Results}
\label{app:results}

\lipsum[61-62]

\subsection{Extended Performance Analysis}
\label{app:extended-performance}

\lipsum[63]

\begin{table}[htbp]
    \centering
    \caption{Extended performance metrics across all experimental conditions}
    \label{tab:extended-performance}
    \begin{tabular}{lcccccc}
        \toprule
        Dataset & Precision & Recall & F1-Score & AUC & MSE & Time (s) \\
        \midrule
        Financial & 0.94 & 0.93 & 0.93 & 0.97 & 0.023 & 12.3 \\
        Biological & 0.91 & 0.90 & 0.91 & 0.95 & 0.031 & 18.7 \\
        Social & 0.93 & 0.92 & 0.93 & 0.96 & 0.027 & 15.2 \\
        \bottomrule
    \end{tabular}
\end{table}

\lipsum[64]

\subsection{Error Analysis}
\label{app:errors}

\lipsum[65-66]

%====================================================================
% REFERENCES
%====================================================================
\bibliographystyle{apalike} %Please change based on journal.
%\bibliographystyle{unsrtnat}
% \bibliography{references} % Uncomment and provide your .bib file

% Manual bibliography for template demonstration
\begin{thebibliography}{99}

\bibitem{example2023}
A.~B. Example and C.~D. Author, ``Advanced techniques in complex systems modeling: A comprehensive review,'' \textit{Journal of Complex Systems Research}, vol.~45, no.~3, pp.~123--145, 2023.

\bibitem{another2022}
E.~F. Another, M.~G. Researcher, and P.~Q. Scholar, ``Machine learning approaches for large-scale data analysis,'' in \textit{Proceedings of the International Conference on Computational Methods}, New York, NY, USA, 2022, pp.~67--89.

\bibitem{foundational2020}
G.~H. Foundational, \textit{Statistical Methods in Complex Systems: Theory and Applications}, 3rd ed. Academic Press, 2020.

\bibitem{recent2024}
L.~M. Recent and N.~O. Contemporary, ``Optimization algorithms for high-dimensional problems,'' \textit{Computational Optimization Letters}, vol.~12, no.~1, pp.~45--62, 2024.

\bibitem{benchmark2021}
R.~S. Benchmark, T.~U. Validation, and V.~W. Testing, ``Benchmark datasets for complex systems analysis,'' \textit{Data Science Repository}, vol.~8, pp.~234--251, 2021.

\end{thebibliography}

%====================================================================
% AUTHOR BIOGRAPHIES (if required by journal)
%====================================================================
\section*{Author Biographies}

\textbf{John A. Smith} is an Assistant Professor in the Department of Computer Science at the University of Excellence. He received his Ph.D. from MIT in 2018. His research interests include machine learning, optimization, and complex systems.

\textbf{Jane B. Doe} is a Ph.D. candidate at the University of Excellence. She received her M.S. from Stanford University in 2020. Her research focuses on computational methods for biological systems.

\textbf{Robert C. Johnson} is a Senior Research Scientist at the Institute for Computational Research. He received his Ph.D. from Carnegie Mellon University in 2010. His research interests include large-scale data analysis and distributed computing.


\end{document}
